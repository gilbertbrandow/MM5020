\documentclass[12pt]{article}
\usepackage{amsmath,amsthm,amssymb}
\title{Abstract Algebra - Homework 1}
\author{Simon Gustafsson}
\date{}

\begin{document}
\maketitle
\section*{Problem 1}

Let \( k = \mathbb{F}_7 = \mathbb{Z}/7\mathbb{Z} \). We define \( \mathrm{GL}_2(k) \) as the group of all invertible \( 2 \times 2 \) matrices with entries in the field \( k \). In other words,
\[
\mathrm{GL}_2(k) = \left\{ \begin{pmatrix} a & b \\ c & d \end{pmatrix} \in M_2(k) \ \middle| \ ad - bc \neq 0 \right\}.
\]

\subsection*{\(\mathrm{GL}_2(k)\) Forms a Group}

Let \( A, B \in \mathrm{GL}_2(k) \). Then the product \( AB \) is defined as:
\[
AB = \begin{pmatrix}
ae + bg & af + bh \\
ce + dg & cf + dh
\end{pmatrix},
\]
where all entries involve only addition and multiplication of elements in \( k \). Since \( k \) is a field, it is closed under addition and multiplication, so all entries of \( AB \) lie in \( k \). Furthermore, \( \det(AB) = \det(A)\det(B) \neq 0 \), so \( AB \in \mathrm{GL}_2(k) \). Thus, \( \mathrm{GL}_2(k) \) is closed under matrix multiplication.

Since matrix multiplication is associative, the operation on \( \mathrm{GL}_2(k) \) is associative.

The identity element in \( \mathrm{GL}_2(k) \) is the identity matrix
\[
I = \begin{pmatrix} 1 & 0 \\ 0 & 1 \end{pmatrix}.
\]
For any \( A \in \mathrm{GL}_2(k) \), we have \( AI = IA = A \), and since the entries \( 1 \) and \( 0 \) are in \( k \), we conclude \( I \in \mathrm{GL}_2(k) \).

Since every element of \( \mathrm{GL}_2(k) \) is invertible by definition, the inverse of
\[
A = \begin{pmatrix} a & b \\ c & d \end{pmatrix} \in \mathrm{GL}_2(k)
\]
is given by:
\[
A^{-1} = \frac{1}{\det(A)} \begin{pmatrix} d & -b \\ -c & a \end{pmatrix}.
\]
Since \( a, b, c, d \in k \), and \( \det(A) = ad - bc \neq 0 \), we have \( \frac{1}{\det(A)} \in k \), because \( k \) is a field. As \( k \) is closed under addition, subtraction, and multiplication, all entries of \( A^{-1} \) lie in \( k \). Hence, \( A^{-1} \in \mathrm{GL}_2(k) \).

Therefore, \( \mathrm{GL}_2(k) \) satisfies the group axioms under matrix multiplication and is a group.

\subsection*{The Center of \(\mathrm{GL}_2(k)\)}

Let \( A = \begin{pmatrix} a & b \\ c & d \end{pmatrix} \in \mathrm{GL}_2(k) \). Suppose \( A \in Z(\mathrm{GL}_2(k)) \). Then for all \( B = \begin{pmatrix} e & f \\ g & h \end{pmatrix} \in \mathrm{GL}_2(k) \), we must have
\[
AB = BA.
\]

Compute both sides:
\[
AB = \begin{pmatrix} ae + bg & af + bh \\ ce + dg & cf + dh \end{pmatrix}, \quad
BA = \begin{pmatrix} ea + fc & eb + fd \\ ga + hc & gb + hd \end{pmatrix}.
\]

For these to be equal for all \( e, f, g, h \in k \), compare the corresponding entries:

\begin{align*}
ae + bg &= ea + fc &\Rightarrow& \quad b = c = 0 \\
af + bh &= eb + fd &\Rightarrow& \quad a = d, \text{ using } b = 0 \\
ce + dg &= ga + hc &\Rightarrow& \quad c = b = 0 \\
cf + dh &= gb + hd &\Rightarrow& \quad d = a
\end{align*}

Hence, \( A \) must be of the form:
\[
A = \begin{pmatrix} \lambda & 0 \\ 0 & \lambda \end{pmatrix} = \lambda I,
\]
where \( \lambda \in k^\times = k \setminus \{0\} \) (since \( A \) must be invertible). Thus,
\[
Z(\mathrm{GL}_2(\mathbb{F}_7)) = \left\{ \lambda I \mid \lambda \in k^\times \right\}.
\]

This center consists of 6 elements and forms an abelian subgroup of \( \mathrm{GL}_2(\mathbb{F}_7) \).

\subsection*{Group Action on \(\mathbb{P}^1(k)\)}

We define the projective line over \( k \) as \( \mathbb{P}^1(k) = k \cup \{\infty\} \). Let \( A = \begin{pmatrix} a & b \\ c & d \end{pmatrix} \in \mathrm{GL}_2(k) \). Then the action of \( \mathrm{GL}_2(k) \) on \( \mathbb{P}^1(k) \) is given by the formula:
\[
A \cdot z =
\begin{cases}
\displaystyle \frac{az + b}{cz + d}, & \text{if } z \in k \text{ and } cz + d \neq 0, \\
\infty, & \text{if } z \in k \text{ and } cz + d = 0, \\
\displaystyle \frac{a}{c}, & \text{if } z = \infty \text{ and } c \neq 0, \\
\infty, & \text{if } z = \infty \text{ and } c = 0.
\end{cases}
\]

We verify that this defines a group action.

Let \( I = \begin{pmatrix} 1 & 0 \\ 0 & 1 \end{pmatrix} \). Then for all \( z \in \mathbb{P}^1(k) \),
\[
I \cdot z = \frac{1 \cdot z + 0}{0 \cdot z + 1} = \frac{z}{1} = z, \quad \text{and} \quad I \cdot \infty = \frac{1}{0} := \infty.
\]
So the identity acts as the identity function.

Let \( A = \begin{pmatrix} a & b \\ c & d \end{pmatrix} \) and \( B = \begin{pmatrix} e & f \\ g & h \end{pmatrix} \) be in \( \mathrm{GL}_2(k) \), and let \( z \in \mathbb{P}^1(k) \). Then:
\[
B \cdot z = \frac{ez + f}{gz + h}, \quad \text{and} \quad A \cdot (B \cdot z) = \frac{a \cdot \left( \frac{ez + f}{gz + h} \right) + b}{c \cdot \left( \frac{ez + f}{gz + h} \right) + d}.
\]
Simplifying this gives:
\[
A \cdot (B \cdot z) = \frac{(aez + af + bgz + bh)}{(cez + cf + dgz + dh)} = \frac{(ae + bg)z + (af + bh)}{(ce + dg)z + (cf + dh)},
\]
which is the action of the product matrix \( AB \) on \( z \):
\[
(AB) \cdot z.
\]
Thus, the compatibility condition holds.

\medskip

Therefore, the formula defines a group action of \( \mathrm{GL}_2(k) \) on \( \mathbb{P}^1(k) \).

\subsection*{Transitivity and the Stabilizer of 0}

Let \( z \in \mathbb{P}^1(k) \) be arbitrary. We want to find a matrix \( A \in \mathrm{GL}_2(k) \) such that \( A \cdot 0 = z \). Let
\[
A = \begin{pmatrix} a & b \\ c & d \end{pmatrix} \in \mathrm{GL}_2(k).
\]
Then the action on \( 0 \) is given by
\[
A \cdot 0 = \frac{b}{d},
\]
provided \( d \neq 0 \). For any \( z \in k \), we can take \( b = z \) and \( d = 1 \), with \( a \in k^\times \), \( c \in k \) arbitrary, such that \( \det(A) = ad - bc \neq 0 \).

To obtain \( \infty \), we require \( d = 0 \), in which case the formula becomes
\[
A \cdot 0 = \frac{b}{d} = \infty,
\]
provided \( b \neq 0 \). We can choose \( a, c \in k \) arbitrarily so that \( \det(A) = -bc \neq 0 \), ensuring \( A \in \mathrm{GL}_2(k) \).

Hence, for any \( z \in \mathbb{P}^1(k) \), there exists a matrix \( A \in \mathrm{GL}_2(k) \) such that \( A \cdot 0 = z \). Therefore, the action is transitive.

We require \( A \cdot 0 = 0 \), so \( \frac{b}{d} = 0 \Rightarrow b = 0 \). Therefore, the stabilizer of 0 is the set of all invertible lower triangular matrices:
\[
B = \left\{ \begin{pmatrix} a & 0 \\ c & d \end{pmatrix} \in \mathrm{GL}_2(k) \ \middle| \ a, d \in k^\times, \ c \in k \right\}.
\]

\subsection*{Kernel of the Action}

The kernel of the action can be defined as follows,
\[
\ker = \left\{ A \in \mathrm{GL}_2(k) \mid A \cdot z = z \text{ for all } z \in \mathbb{P}^1(k) \right\}.
\]

Then, for all \( z \in k \),
\begin{align*}
A \cdot z &= \frac{az + b}{cz + d} = z \\
&\implies az + b = z(cz + d) = cz^2 + dz \\
&\implies cz^2 + (d - a)z - b = 0 \\
&\implies \quad c = 0, \quad d = a, \quad b = 0.
\end{align*}

Therefore,
\[
A = \begin{pmatrix} a & 0 \\ 0 & a \end{pmatrix} = aI, \quad \text{with } a \in k^\times.
\]
These are exactly the scalar matrices which form the center. Hence, the kernel of the action is the center of \( \mathrm{GL}_2(k) \).

\subsection*{Intersection of Conjugates of the Stabilizer}

Let \( B \subseteq \mathrm{GL}_2(k) \) be the stabilizer of \( 0 \in \mathbb{P}^1(k) \). We claim that
\[
\bigcap_{g \in \mathrm{GL}_2(k)} gBg^{-1} = Z(\mathrm{GL}_2(k)).
\]

This follows from the general fact that, if a group \( G \) acts transitively on a set \( X \), then the intersection of all conjugates of the stabilizer \( G_x \) is equal to the kernel of the associated homomorphism:
\[
\bigcap_{g \in G} gG_xg^{-1} = \ker(G \to \mathrm{Sym}(X)).
\]

In our case, \( G = \mathrm{GL}_2(k) \), \( X = \mathbb{P}^1(k) \). We have already shown that the action is transitive, and that the kernel of the action is the center:
\[
\ker = Z(\mathrm{GL}_2(k)).
\]

Therefore,
\[
\bigcap_{g \in \mathrm{GL}_2(k)} gBg^{-1} = Z(\mathrm{GL}_2(k)).
\]

\section*{Problem 2}
Let \( X \) be the set of all \( k \)-element subsets of \( \{1, \dots, n\} \), and let \( S_n \) act on \( X \) by
\[
\sigma \cdot E := \{ \sigma(e) \mid e \in E \}.
\]

Fix the subset \( A = \{1, \dots, k\} \in X \). The stabilizer of \( A \) consists of all permutations in \( S_n \) that fix \( A \) setwise. These are exactly the permutations that act as an element of \( S_k \) on the set \( \{1, \dots, k\} \), and as an element of \( S_{n-k} \) on its complement \( \{k+1, \dots, n\} \), independently. Thus,
\[
\mathrm{Stab}_{S_n}(A) \cong S_k \times S_{n-k}.
\]

By the Orbit-Stabilizer Theorem, the number of \( k \)-element subsets is
\[
|X| = [S_n : \mathrm{Stab}_{S_n}(A)] = \frac{n!}{k!(n - k)!} = \binom{n}{k}.
\]

\section*{Problem 3}
\subsection*{Group action}

Let \( G \) be a finite group and \( H \subset G \) a subgroup. Define a map
\[
\phi  : H \times (G/H) \to G/H, \quad (h, gH) \mapsto (hg)H.
\]

To verify that \( \phi \) defines a group action, we check the two axioms. For all \( gH \in G/H \), 
\[e \cdot gH = (eg)H = gH.\]
For all \( h_1, h_2 \in H \) and \( gH \in G/H \), 
\[h_1 \cdot (h_2 \cdot gH) = h_1 \cdot (h_2 g H) = (h_1 h_2 g)H = ((h_1 h_2) \cdot gH).\]

Hence, the identity and compatibility axioms are satisfied. Since we are viewing \( G/H \) purely as a set of left cosets (not as a quotient group), this confirms that \( \phi \) defines a group action of \( H \) on the set \( G/H \).

\subsection*{Trivial action implies normality}

Suppose the action defined by
\[
\phi : H \times (G/H) \to G/H, \quad (h, gH) \mapsto (hg)H
\]
is trivial. That means for all \( h \in H \) and \( gH \in G/H \), we have:
\[
\phi(h, gH) = gH \quad \Rightarrow \quad (hg)H = gH.
\]

This implies:
\[
hg \in gH \quad \Rightarrow \quad g^{-1} h g \in H \quad \text{for all } h \in H, g \in G.
\]

Therefore, \( H \) is closed under conjugation by elements of \( G \), i.e., \( gHg^{-1} \subseteq H \). Since \( g \in G \) was arbitrary, it follows that \(H\) is a normal subgroup of \(G\).

\subsection*{The action is not transitive}

Assume \( [G : H] = p \), where \( p \) is the smallest prime dividing \( |G| \).

Suppose, for contradiction, that the action is transitive. Then the orbit of any element \( gH \in G/H \) under the action of \( H \) is the entire set \( G/H \), which has \( p \) elements.

But the size of any orbit under a group action divides the order of the acting group and in this case, \( |H| = |G|/p \).

Therefore, the size of the orbit must divide \( |H| = |G|/p \), but it is equal to \( p \). Since \( p \) is the smallest prime dividing \( |G| \), it does not divide \( |G|/p \), and thus does not divide \( |H| \).

This is a contradiction. Hence, the action cannot be transitive.

\end{document}
