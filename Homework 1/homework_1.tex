\documentclass[12pt]{article}
\usepackage{amsmath,amsthm,amssymb}
\usepackage{enumitem}
\title{Abstract Algebra - Homework 1}
\author{Simon Gustafsson}
\date{}

\begin{document}
\maketitle
\section*{Problem 1}

Let \( k = \mathbb{F}_7 = \mathbb{Z}/7\mathbb{Z} \). We define \( \mathrm{GL}_2(k) \) as the group of all invertible \( 2 \times 2 \) matrices with entries in the field \( k \). In other words,
\[
\mathrm{GL}_2(k) = \left\{ \begin{pmatrix} a & b \\ c & d \end{pmatrix} \in M_2(k) \ \middle| \ ad - bc \neq 0 \right\}.
\]

\begin{enumerate}[label=(\arabic*)] 

\item 
Let \( A, B \in \mathrm{GL}_2(k) \). Then the product \( AB \) is defined as:
\[
AB = \begin{pmatrix}
ae + bg & af + bh \\
ce + dg & cf + dh
\end{pmatrix},
\]
where all entries involve only addition and multiplication of elements in \( k \). Since \( k \) is a field, it is closed under addition and multiplication, so all entries of \( AB \) lie in \( k \). Furthermore, \( \det(AB) = \det(A)\det(B) \neq 0 \), so \( AB \in \mathrm{GL}_2(k) \). Thus, \( \mathrm{GL}_2(k) \) is closed under matrix multiplication.

Since matrix multiplication is associative, the operation on \( \mathrm{GL}_2(k) \) is associative.

The identity element in \( \mathrm{GL}_2(k) \) is the identity matrix
\[
I = \begin{pmatrix} 1 & 0 \\ 0 & 1 \end{pmatrix}.
\]
For any \( A \in \mathrm{GL}_2(k) \), we have \( AI = IA = A \), and since the entries \( 1 \) and \( 0 \) are in \( k \), we conclude \( I \in \mathrm{GL}_2(k) \).

Since every element of \( \mathrm{GL}_2(k) \) is invertible by definition, the inverse of
\[
A = \begin{pmatrix} a & b \\ c & d \end{pmatrix} \in \mathrm{GL}_2(k)
\]
is given by:
\[
A^{-1} = \frac{1}{\det(A)} \begin{pmatrix} d & -b \\ -c & a \end{pmatrix}.
\]
Since \( a, b, c, d \in k \), and \( \det(A) = ad - bc \neq 0 \), we have \( \frac{1}{\det(A)} \in k \), because \( k \) is a field. As \( k \) is closed under addition, subtraction, and multiplication, all entries of \( A^{-1} \) lie in \( k \). Hence, \( A^{-1} \in \mathrm{GL}_2(k) \).

Therefore, \( \mathrm{GL}_2(k) \) satisfies the group axioms under matrix multiplication and is a group.

Let \( A = \begin{pmatrix} a & b \\ c & d \end{pmatrix} \in \mathrm{GL}_2(k) \). Suppose \( A \in Z(\mathrm{GL}_2(k)) \). Then for all \( B = \begin{pmatrix} e & f \\ g & h \end{pmatrix} \in \mathrm{GL}_2(k) \), we must have
\[
AB = BA.
\]

Where,
\[
AB = \begin{pmatrix} ae + bg & af + bh \\ ce + dg & cf + dh \end{pmatrix}, \quad
BA = \begin{pmatrix} ea + fc & eb + fd \\ ga + hc & gb + hd \end{pmatrix}.
\]

For these to be equal \( \forall e, f, g, h \in k \), the corresponding entries must be;
\[
\begin{aligned}
ae + bg &= ea + fc &\Longrightarrow&\; bg = fc \Longrightarrow b = c = 0,\\
af + bh &= fd         &\Longrightarrow&\; af = fd \Longrightarrow a = d.\\
\end{aligned}
\]
Hence, \( A \) must be of the form:
\[
A = \begin{pmatrix} \lambda & 0 \\ 0 & \lambda \end{pmatrix} = \lambda I,
\]
where \( \lambda \in k^\times = k \setminus \{0\} \) as a consequence of the fact that \( A \) is invertible. Thus,
\[
Z(\mathrm{GL}_2(\mathbb{F}_7)) = \left\{ \lambda I \mid \lambda \in k^\times \right\}.
\]

\item
For the identity matrix \(I=\begin{pmatrix}1&0\\0&1\end{pmatrix}\) we have
\[
I\cdot z = \frac{1\cdot z+0}{0\cdot z+1}=z, \quad \forall z\in k
\qquad
I\cdot\infty = \infty.
\]
Thus \(I\) acts as the identity on \(\mathbb{P}^1(k)\).

For
\(A=\begin{pmatrix}a&b\\ c&d\end{pmatrix},
 \;B=\begin{pmatrix}e&f\\ g&h\end{pmatrix}\in\mathrm{GL}_2(k)\),
\[
\begin{aligned}
A\cdot(B\cdot\! z)
     &=\frac{a\frac{ez+f}{gz+h}+b}{\,c\frac{ez+f}{gz+h}+d\,} \\
     &=\frac{(ae+bg)z+(af+bh)}{(ce+dg)z+(cf+dh)} \\
     &=\begin{pmatrix}ae+bg&af+bh\\ ce+dg&cf+dh\end{pmatrix} \cdot z \\
     &= (AB) \cdot z
\end{aligned}
\]

For \(z=\infty\) we have \(B\cdot\infty = e/g\) if \(g\neq0\) and \(\infty\) if \(g=0\);
in either case the same calculation gives \(A\cdot(B\cdot\infty)=(AB)\cdot\infty\). Thus \(A\cdot(B\cdot z)=(AB)\cdot z, \forall z\in\mathbb P^{1}(k)\). This verifies that the
formula defines a group action of \(\mathrm{GL}_2(k)\) on \(\mathbb{P}^1(k)\).
\item

Let \( z \in \mathbb{P}^1(k) \) be arbitrary and let
\(
A = \begin{pmatrix} a & b \\ c & d \end{pmatrix} \in \mathrm{GL}_2(k).
\)
Then the action on \( 0 \) is given by
\[
A \cdot 0 = \frac{b}{d},
\]
provided \( d \neq 0 \). For any \( z \in k \), we can take \( b = z \) and \( d = 1 \), with \( a \in k^\times \), \( c \in k \) arbitrary, such that \( \det(A) = ad - bc \neq 0 \).

To obtain \( \infty \), we require \( d = 0 \), in which case the formula becomes
\[
A \cdot 0 = \frac{b}{d} = \infty,
\]
provided \( b \neq 0 \). We can choose \( a, c \in k \) arbitrarily so that \( \det(A) = -bc \neq 0 \), ensuring \( A \in \mathrm{GL}_2(k) \).

Hence, for any \( z \in \mathbb{P}^1(k) \), there exists a matrix \( A \in \mathrm{GL}_2(k) \) such that \( A \cdot 0 = z \). Therefore, the action is transitive.

We require \( A \cdot 0 = 0 \), so \( \frac{b}{d} = 0 \Rightarrow b = 0 \). Therefore, the stabilizer of 0 is the set of all invertible lower triangular matrices:
\[
B = \left\{ \begin{pmatrix} a & 0 \\ c & d \end{pmatrix} \in \mathrm{GL}_2(k) \ \middle| \ a, d \in k^\times, \ c \in k \right\}.
\]

\item

The kernel of the action can be defined as follows,
\[
\ker = \left\{ A \in \mathrm{GL}_2(k) \mid A \cdot z = z, \forall z \in \mathbb{P}^1(k) \right\}.
\]

Then, \(\forall z \in k \),
\begin{align*}
A \cdot z = \frac{az + b}{cz + d} = z &\implies az + b = z(cz + d) = cz^2 + dz \\
&\implies cz^2 + (d - a)z - b = 0 \\
&\implies \quad c = 0, \quad d = a, \quad b = 0.
\end{align*}

Therefore,
\[
A = \begin{pmatrix} a & 0 \\ 0 & a \end{pmatrix} = aI, \quad \text{with } a \in k^\times.
\]
These are exactly the scalar matrices which form the center. Hence, the kernel of the action is the center of \( \mathrm{GL}_2(k) \).

\item

Let \( B \subseteq \mathrm{GL}_2(k) \) be the stabilizer of \( 0 \in \mathbb{P}^1(k) \). We claim that
\[
\bigcap_{g \in \mathrm{GL}_2(k)} gBg^{-1} = Z(\mathrm{GL}_2(k)).
\]

This follows from the general fact that, if a group \( G \) acts transitively on a set \( X \), then the intersection of all conjugates of the stabilizer \( G_x \) is equal to the kernel of the associated homomorphism:
\[
\bigcap_{g \in G} gG_xg^{-1} = \ker(G \to \mathrm{Sym}(X)).
\]

In our case, \( G = \mathrm{GL}_2(k) \), \( X = \mathbb{P}^1(k) \). We have already shown that the action is transitive, and that the kernel of the action is the center:
\[
\ker = Z(\mathrm{GL}_2(k)).
\]

Therefore,
\[
\bigcap_{g \in \mathrm{GL}_2(k)} gBg^{-1} = Z(\mathrm{GL}_2(k)).
\]

\end{enumerate}

\section*{Problem 2}
\begin{enumerate}[label=(\arabic*)] 

\item Let \( X \) be the set of all \( k \)-element subsets of \( \{1, \dots, n\} \), and let \( S_n \) act on \( X \) by
\[
\sigma \cdot E := \{ \sigma(e) \mid e \in E \}.
\]

Fix the subset \( A = \{1, \dots, k\} \in X \). The stabilizer of \( A \) consists of all permutations in \( S_n \) that fix \( A \) setwise. These are exactly the permutations that act as an element of \( S_k \) on the set \( \{1, \dots, k\} \), and as an element of \( S_{n-k} \) on its complement \( \{k+1, \dots, n\} \), independently. Thus,
\[
\mathrm{Stab}_{S_n}(A) \cong S_k \times S_{n-k}.
\]

\item By the Orbit-Stabilizer Theorem, the number of \( k \)-element subsets is
\[
|X| = [S_n : \mathrm{Stab}_{S_n}(A)] = \frac{n!}{k!(n - k)!} = \binom{n}{k}.
\]
\end{enumerate}

\section*{Problem 3}

\begin{enumerate}[label=(\arabic*)] 

\item To verify that \( \phi \) defines a group action, we check the two axioms. For all \( gH \in G/H \), 
\[e \cdot gH = (eg)H = gH.\]
For all \( h_1, h_2 \in H \) and \( gH \in G/H \), 
\[h_1 \cdot (h_2 \cdot gH) = h_1 \cdot (h_2 g H) = (h_1 h_2 g)H = ((h_1 h_2) \cdot gH).\]

Hence, the identity and compatibility axioms are satisfied. Since we are viewing \( G/H \) purely as a set of left cosets, this confirms that \( \phi \) defines a group action of \( H \) on the set \( G/H \).

\item

Suppose $\phi$ is trivial. That means for all \( h \in H \) and \( gH \in G/H \), we have:
\[
\phi(h, gH) = gH \quad \Rightarrow \quad (hg)H = gH.
\]

This implies:
\[
hg \in gH \quad \Rightarrow \quad g^{-1} h g \in H \quad \text{for all } h \in H, g \in G.
\]

Therefore, \( H \) is closed under conjugation by elements of \( G \), i.e., \( gHg^{-1} \subseteq H \). 
Since \( g \in G \) was arbitrary, it follows that \(H\) is a normal subgroup of \(G\).

\item

Assume \( [G : H] = p \), where \( p \) is the smallest prime dividing \( |G| \).

Suppose that the action is transitive. Then the orbit of any element \( gH \in G/H \) under the action of \( H \) is the entire set \( G/H \), which has \( p \) elements.

But the size of any orbit under a group action divides the order of the acting group and in this case, \( |H| = |G|/p \).

Therefore, the size of the orbit must divide \( |H| = |G|/p \), but it is equal to \( p \). Since \( p \) is the smallest prime dividing \( |G| \), it does not divide \( |G|/p \), and thus does not divide \( |H| \).

This is a contradiction. Hence, the action cannot be transitive.

\item

From the theory of group actions, we have the orbit decomposition:
\[
|G/H| = \sum_{i=1}^r |\mathcal{O}_i|,
\]
where \( \mathcal{O}_i \) are the distinct orbits of the action of \( H \) on the set \( G/H \).

Since \( |G/H| = p \), a prime number, the possible orbit decompositions are very limited. One possibility is that there is a single orbit of size \( p \), but this would imply that the action is transitive, which we ruled out in the previous section. Another possibility is that there is one orbit of size 1 and one of size \( p - 1 \), but the size of each orbit must divide \( |H| = |G|/p \). Since \( p \) is the smallest prime dividing \( |G| \), it does not divide \( |G|/p \), and therefore cannot divide any orbit size greater than 1.

The only remaining possibility is that all orbits have size 1. That means:
\[
(hg)H = gH \quad \forall h \in H,\, gH \in G/H,
\]
which implies \( hg \in gH \Rightarrow g^{-1}hg \in H \), i.e., \( H \) is invariant under conjugation by all \( g \in G \). Therefore, \( H \trianglelefteq G \).
\end{enumerate}

\section*{Problem 4}

\begin{enumerate}[label=(\arabic*)] 

\item

By definition, \( V_4 \) contains the identity element. To show that \( V_4 \) is normal in \( S_4 \), we verify that it is invariant under conjugation. 
For any \( \sigma \in S_4 \) and \( v \in V_4 \), we have \( \sigma v \sigma^{-1} \) is again a product of two disjoint transpositions. 
Since there are exactly three such elements in \( S_4 \), and they form a conjugacy class, it follows that conjugation by any \( \sigma \in S_4 \) sends elements of \( V_4 \) to other elements in \( V_4 \).

Hence, \( V_4 \) is closed under conjugation, and we conclude that \( V_4 \trianglelefteq S_4 \).

\item 
Consider the subgroup \( H = \langle (12), (123) \rangle \subset S_4 \). This subgroup permutes the elements \( \{1, 2, 3\} \) and fixes 4, so it is isomorphic to \( S_3 \). We identify \( S_3 \) with this subgroup \( H \).

\medskip

We define the map
\[
f : S_3 \to S_4 / V_4, \quad f(\sigma) = \sigma V_4.
\]

First, it is well-defined: each \( \sigma \in S_3 \) is interpreted as an element of \( H \subset S_4 \), and the left coset \( \sigma V_4 \) is a valid element of the quotient \( S_4 / V_4 \). The map is injective because \( H \cap V_4 = \{ \text{Id} \} \), so no two distinct elements of \( H \) lie in the same coset. It is surjective since \( H \) has 6 elements and \( |S_4 / V_4| = 6 \), meaning the image of \( f \) exhausts all cosets. Finally, \( f \) is a homomorphism: for all \( \sigma_1, \sigma_2 \in S_3 \), we have
\[
f(\sigma_1 \sigma_2) = \sigma_1 \sigma_2 V_4 = \sigma_1 V_4 \cdot \sigma_2 V_4 = f(\sigma_1) f(\sigma_2).
\]
Thus, \( f \) is a bijective homomorphism and therefore an isomorphism.

\item From group theory, the normal subgroups of \( S_4 \) are:
\[
\{ \text{Id} \}, \quad V_4, \quad A_4, \quad S_4.
\]
Here \(A_4\) denotes the alternating group on four letters, the set of all even permutations, so it is normal in \(S_4\). Moreover, because each element of \(V_4\) is a product of two transpositions (an even permutation), we have \(V_4 \subset A_4\).
The subgroups that contain \( V_4 \) are \(V_4, A_4, S_4.\) These are all normal in \( S_4 \), and there are no other normal subgroups strictly between \( V_4 \) and \( S_4 \). 
Hence, the normal subgroups of \( S_4 \) containing \( V_4 \) are \(V_4, A_4, S_4.\) 
\end{enumerate}
\section*{Problem 5}
\begin{enumerate}[label=(\arabic*)] 

\item
Since \(5<7\) and \(5\nmid(7-1)=6\), the $pq$-order theorem implies that every group of order \(pq\) with those divisibility conditions is cyclic.  Hence
\[
G \cong \mathbb Z_{35}.
\]
For a cyclic domain, a homomorphism is completely determined by the image of a generator \(g\in G\).  Suppose
\[
\varphi : G \longrightarrow S_{3},\qquad \varphi(g)=x.
\]
Then \(\mathrm{ord}(x)\) must divide \(\mathrm{ord}(g)=35\).

The possible element orders in \(S_{3}\) are \(1,2,3\), and among these only \(1\) divides \(35\).  Therefore \(x\) must be the identity permutation; consequently \(\varphi\) sends every element of \(G\) to the identity in \(S_{3}\).  This is the trivial homomorphism, and no other homomorphism can exist.

\item

An action of \(G\) on the set \(E=\{1,2,3\}\) is equivalent to a group homomorphism
\[
\rho : G \longrightarrow \operatorname{Sym}(E)=S_{3}.
\]
In the previous section we proved that, for a group \(G\) of order \(35\), the only homomorphism \(G\to S_{3}\) is the trivial one.  

Hence there is exactly 1 action of \(G\) on \(E\), namely the trivial action \(g\cdot x = x\) for all \(g\in G\) and \(x\in E\).

\item
Because \(|E| = 3\), the possible orbit decompositions are
\(3\), \(2+1\), or \(1+1+1\).
A single orbit of size \(3\) contradicts non-transitivity, while
three singleton orbits give the trivial action.
Hence \(E\) must split into one orbit of size \(2\) and one of size \(1\).

Let \(y\) lie in the two-element orbit and \(x\) in the singleton orbit.
By the Orbit-Stabiliser Theorem,
\[
|G| = |G\!\cdot\!y|\;|G_{y}| = 2\,|G_{y}|,\qquad
|G| = |G\!\cdot\!x|\;|G_{x}| = 1\cdot |G_{x}|.
\]
Thus
\[
|G_{x}| = |G|,\qquad |G_{y}| = \frac{|G|}{2}.
\]
Because \(|G|\) is even, \(|G_{y}|\) is an integer, and
the stabiliser of \(y\) has index \(2\) in \(G\).
The action is genuinely non-trivial, because the two-element orbit is not fixed point-wise. There exists some \(g\in G\) with \(g\!\cdot\!y \neq y\).

Therefore the only non-trivial, non-transitive action has two orbits
of sizes \(2\) and \(1\). The singleton orbit is fixed point-wise, while
the stabiliser of each point in the two-element orbit has index~\(2\) in
\(G\).

\item
Suppose \(G\) has odd order and acts on \(E=\{1,2,3\}\).
If the action were non-transitive and non-trivial, previous section implies that \(E\) would split into
one orbit of size \(2\) and one of size \(1\).
Choose \(y\) in the two-element orbit.
By the Orbit-Stabiliser Theorem,
\[
|G| \;=\; |G\!\cdot\!y|\,|G_{y}| \;=\; 2\,|G_{y}|.
\]
Hence \(|G_{y}| = |G|/2\), but \(|G|\) is odd, so \(|G|/2\notin\mathbb Z\) which is a contradiction.

Consequently an odd-order group cannot have a non-transitive, non-trivial action on a three-element set.
\end{enumerate}

\end{document}
