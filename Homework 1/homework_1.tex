\documentclass[12pt]{article}
\usepackage{amsmath,amsthm,amssymb}
\title{Abstract Algebra - Homework 1}
\author{Simon Gustafsson}
\date{}

\begin{document}
\maketitle
\section{Group Action on the Projective Line}

Let \( k = \mathbb{F}_7 = \mathbb{Z}/7\mathbb{Z} \). We define \( \mathrm{GL}_2(k) \) as the group of all invertible \( 2 \times 2 \) matrices with entries in the field \( k \). In other words,
\[
\mathrm{GL}_2(k) = \left\{ \begin{pmatrix} a & b \\ c & d \end{pmatrix} \in M_2(k) \ \middle| \ ad - bc \neq 0 \right\}.
\]

\subsection*{\(\mathrm{GL}_2(k)\) Forms a Group}

Let \( A, B \in \mathrm{GL}_2(k) \). Then the product \( AB \) is defined as:
\[
AB = \begin{pmatrix}
ae + bg & af + bh \\
ce + dg & cf + dh
\end{pmatrix},
\]
where all entries involve only addition and multiplication of elements in \( k \). Since \( k \) is a field, it is closed under addition and multiplication, so all entries of \( AB \) lie in \( k \). Furthermore, \( \det(AB) = \det(A)\det(B) \neq 0 \), so \( AB \in \mathrm{GL}_2(k) \). Thus, \( \mathrm{GL}_2(k) \) is closed under matrix multiplication.

Since matrix multiplication is associative, the operation on \( \mathrm{GL}_2(k) \) is associative.

The identity element in \( \mathrm{GL}_2(k) \) is the identity matrix
\[
I = \begin{pmatrix} 1 & 0 \\ 0 & 1 \end{pmatrix}.
\]
For any \( A \in \mathrm{GL}_2(k) \), we have \( AI = IA = A \), and since the entries \( 1 \) and \( 0 \) are in \( k \), we conclude \( I \in \mathrm{GL}_2(k) \).

Since every element of \( \mathrm{GL}_2(k) \) is invertible by definition, the inverse of
\[
A = \begin{pmatrix} a & b \\ c & d \end{pmatrix} \in \mathrm{GL}_2(k)
\]
is given by:
\[
A^{-1} = \frac{1}{\det(A)} \begin{pmatrix} d & -b \\ -c & a \end{pmatrix}.
\]
Since \( a, b, c, d \in k \), and \( \det(A) = ad - bc \neq 0 \), we have \( \frac{1}{\det(A)} \in k \), because \( k \) is a field. As \( k \) is closed under addition, subtraction, and multiplication, all entries of \( A^{-1} \) lie in \( k \). Hence, \( A^{-1} \in \mathrm{GL}_2(k) \).

Therefore, \( \mathrm{GL}_2(k) \) satisfies the group axioms under matrix multiplication and is a group.

\subsection*{The Center of \(\mathrm{GL}_2(k)\)}

Let \( A = \begin{pmatrix} a & b \\ c & d \end{pmatrix} \in \mathrm{GL}_2(k) \). Suppose \( A \in Z(\mathrm{GL}_2(k)) \). Then for all \( B = \begin{pmatrix} e & f \\ g & h \end{pmatrix} \in \mathrm{GL}_2(k) \), we must have
\[
AB = BA.
\]

Compute both sides:
\[
AB = \begin{pmatrix} ae + bg & af + bh \\ ce + dg & cf + dh \end{pmatrix}, \quad
BA = \begin{pmatrix} ea + fc & eb + fd \\ ga + hc & gb + hd \end{pmatrix}.
\]

For these to be equal for all \( e, f, g, h \in k \), compare the corresponding entries:

\begin{align*}
ae + bg &= ea + fc &\Rightarrow& \quad b = c = 0 \\
af + bh &= eb + fd &\Rightarrow& \quad a = d, \text{ using } b = 0 \\
ce + dg &= ga + hc &\Rightarrow& \quad c = b = 0 \\
cf + dh &= gb + hd &\Rightarrow& \quad d = a
\end{align*}

Hence, \( A \) must be of the form:
\[
A = \begin{pmatrix} \lambda & 0 \\ 0 & \lambda \end{pmatrix} = \lambda I,
\]
where \( \lambda \in k^\times = k \setminus \{0\} \) (since \( A \) must be invertible). Thus,
\[
Z(\mathrm{GL}_2(\mathbb{F}_7)) = \left\{ \lambda I \mid \lambda \in k^\times \right\}.
\]

This center consists of 6 elements and forms an abelian subgroup of \( \mathrm{GL}_2(\mathbb{F}_7) \).

\subsection*{Group Action on \(\mathbb{P}^1(k)\)}

We define the projective line over \( k \) as \( \mathbb{P}^1(k) = k \cup \{\infty\} \). Let \( A = \begin{pmatrix} a & b \\ c & d \end{pmatrix} \in \mathrm{GL}_2(k) \). Then the action of \( \mathrm{GL}_2(k) \) on \( \mathbb{P}^1(k) \) is given by the formula:
\[
A \cdot z =
\begin{cases}
\displaystyle \frac{az + b}{cz + d}, & \text{if } z \in k \text{ and } cz + d \neq 0, \\
\infty, & \text{if } z \in k \text{ and } cz + d = 0, \\
\displaystyle \frac{a}{c}, & \text{if } z = \infty \text{ and } c \neq 0, \\
\infty, & \text{if } z = \infty \text{ and } c = 0.
\end{cases}
\]

We verify that this defines a group action.

\paragraph{Identity:}
Let \( I = \begin{pmatrix} 1 & 0 \\ 0 & 1 \end{pmatrix} \). Then for all \( z \in \mathbb{P}^1(k) \),
\[
I \cdot z = \frac{1 \cdot z + 0}{0 \cdot z + 1} = \frac{z}{1} = z, \quad \text{and} \quad I \cdot \infty = \frac{1}{0} := \infty.
\]
So the identity acts as the identity function.

\paragraph{Compatibility:}
Let \( A = \begin{pmatrix} a & b \\ c & d \end{pmatrix} \) and \( B = \begin{pmatrix} e & f \\ g & h \end{pmatrix} \) be in \( \mathrm{GL}_2(k) \), and let \( z \in \mathbb{P}^1(k) \). Then:
\[
B \cdot z = \frac{ez + f}{gz + h}, \quad \text{and} \quad A \cdot (B \cdot z) = \frac{a \cdot \left( \frac{ez + f}{gz + h} \right) + b}{c \cdot \left( \frac{ez + f}{gz + h} \right) + d}.
\]
Simplifying this gives:
\[
A \cdot (B \cdot z) = \frac{(aez + af + bgz + bh)}{(cez + cf + dgz + dh)} = \frac{(ae + bg)z + (af + bh)}{(ce + dg)z + (cf + dh)},
\]
which is the action of the product matrix \( AB \) on \( z \):
\[
(AB) \cdot z.
\]
Thus, the compatibility condition holds.

\medskip

Therefore, the formula defines a group action of \( \mathrm{GL}_2(k) \) on \( \mathbb{P}^1(k) \).


\end{document}
